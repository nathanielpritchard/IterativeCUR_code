
\subsection{Notation}\label{sec:not}
This paper will focus on techniques to construct a rank-$r$ approximation to a matrix $\mat A$. To refer to the $j^{\text{th}}$ column of $\mat A$ we use the notation $\mat A(:, j)$ and to refer to the $i^{\text{th}}$  row of $\mat A$ we use the notation $\mat A(i,:)$. 

Our approach will iteratively form a $\mat C \mat U \mat R$ approximation to $\mat A$ where $\mat C = \mat A(:, J)$, $\mat U = \mat A(I, J)^\dagger$, and $\mat R = \mat A(I, :)$. Here, $I$ is the set of all row indices selected after $k$ iterations, and $J$ represents the set of all column indices selected after $k$ iterations.  Each iteration improves the approximation of iteration $k-1$ by selecting $b=O(1)$ new row indices $I_k$ and column indices $J_k$. We will denote appending the new indices $I_k$ and $J_k$ to $I$ and $J$ with $I = [I, I_k]$ and $J = [J, J_k]$.

The true approximation quality of a CUR approximation can be determined by computing its residual, $\mat S = \mat A- \mat C \mat U \mat R$. In addition to this quantity, we also define two other residuals. The first is the column-residual, $\mat S_{k}^{\rm col}$, which applies a fixed sketching matrix $\mat G \in \real^{c \times m}$, $c = \lfloor 1.1b \rfloor$ where $\lfloor \cdot \rfloor$ is the floor function to the residual to obtain $\mat S_{k}^{\rm col} = \mat G \mat A - \mat G \mat C \mat U \mat R$. The second quantity is the row-residual, $\mat S_{k}^{\rm col}$,  which is the residual at the column indices $J_k$. Specifically, $\mat S_{k}^{\rm row} = \mat A(:, J_k) - \mat C \mat U \mat R(:, J_k)$. We can then use the column residual to measure the quality of a CUR approximation $\rho_k = \|\mat S_{k}^{\rm col}\|_F/\|\mat A\|_F$. We can also determine that a solution is good enough when $\rho_k < \epsilon$, where $\epsilon$ is a user-selected criterion for approximation quality.
