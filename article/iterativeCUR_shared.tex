% SIAM Shared Information Template
% This is information that is shared between the main document and any
% supplement. If no supplement is required, then this information can
% be included directly in the main document.


% Packages and macros go here
\usepackage{lipsum}
\usepackage{amsfonts}
\usepackage{graphicx}
\usepackage{epstopdf}
%\usepackage{algorithmic}
\usepackage{algpseudocode }
\usepackage{booktabs}
\usepackage{amsmath,amssymb}
\usepackage{mathtools}
\usepackage{stmaryrd}
\usepackage{pgfplots}
\usepgfplotslibrary{colorbrewer, groupplots}
\usetikzlibrary{pgfplots.statistics, pgfplots.colorbrewer, external}
\usepackage{pgfplotstable}
\usepackage{filecontents}
\usepackage{longtable}
\usepackage{makecell}

\ifpdf
  \DeclareGraphicsExtensions{.eps,.pdf,.png,.jpg}
\else
  \DeclareGraphicsExtensions{.eps}
\fi

% Personalized commands
\newcommand{\nat}{\mathbb{N}}   
\newcommand{\real}{\mathbb{R}}   
\newcommand{\ex}{\mathbb{E}}   
\newcommand{\pr}{\mathbb{P}}     
\newcommand{\mat}[1]{\boldsymbol{#1}}   
\renewcommand{\vec}[1]{\boldsymbol{#1}} 
\newcommand{\norm}[1]{\left\| {#1} \right\|}

\newcommand{\blkmat}[1]{\begin{bmatrix} #1 \end{bmatrix}}
\newcommand{\twotwo}[4]{\blkmat{#1 & #2 \\ #3 & #4}}
\newcommand{\twoone}[2]{\blkmat{#1 \\ #2}}
\newcommand{\onetwo}[2]{\blkmat{#1 & #2}}
\newcommand{\lowrank}[2]{\left\llbracket #1 \right\rrbracket_{#2}}
\newcommand{\icur}{\cref{alg:ICUR}}
\newcommand{\icurl}{IterativeCUR-LUPP}
\newcommand{\icuro}{IterativeCUR-Osinsky}
\newcommand{\icurq}{IterativeCUR-QRPP}
\newcommand{\curs}{s-LUPP}
\newcommand{\svds}{SVDSketch}
% comment commands
\newcommand{\np}[1]{{\color{cyan} [\textbf{NP:} #1]}}
\newcommand{\yn}[1]{{\color{blue} [\textbf{YN:} #1]}}
\newcommand{\gm}[1]{{\color{blue} [\textbf{GM:} #1]}}
\newcommand{\tp}[1]{{\color{blue} [\textbf{TP:} #1]}}

% commands for plotting
\newcommand{\fopa}{.3}
\newcommand{\dopa}{1}
\newcommand{\boxplotscale}{.7}

% colors for plotting
\definecolor{cSVDs}{RGB}{141, 8, 1}
\definecolor{cLU}{RGB}{227, 178, 60}
\definecolor{cIQR}{RGB}{0, 20, 83}
\definecolor{cILU}{RGB}{12, 155, 107}
\definecolor{cOS}{RGB}{251, 97, 7}
\definecolor{cSQR}{RGB}{67, 97, 238}
\definecolor{cSOS}{RGB}{247, 37, 133}
\definecolor{cSVD}{RGB}{27, 99, 110}
\definecolor{cminus}{RGB}{145, 0, 0}
\definecolor{cpos}{RGB}{79, 120, 59}
\definecolor{c5}{RGB}{191,67,66}
\definecolor{c50}{RGB}{231, 215, 193}
\definecolor{c100}{RGB}{167, 138, 127}
\definecolor{c500}{RGB}{115, 87, 81}
\newcommand{\markersize}{1}

\pgfplotsset{
    numformat single/.style={pgf/number format/#1},
    numformat/.style={numformat single/.list={#1}},
    every axis/.append style={
         width = .5 \textwidth,
        height = .4 \textwidth,
        every tick/.style = {xtick pos = left},
        title style={font=\bfseries},
        label style={font=\small},
        ticklabel style={font=\footnotesize},
        legend style={font=\footnotesize},
        scaled y ticks = false,
    },
    every axis plot/.append style={
    every mark/.append style={solid,  fill opacity = .7, draw opacity = 1},
    scur_line/.style = {mark=triangle, mark size = 3pt, color = cLU, dashed},
    icur_line/.style={mark=square, mark size = 2pt, color = cILU, densely dotted},
    svds_line/.style={mark=o, mark size = 2pt, color = cSVDs},
    svd_line/.style={mark=pentagon, mark size = 3pt, color = cSVD, dash dot dot},
    qr_line/.style={mark=diamond, mark size = 3pt, color = cIQR},
    os_line/.style={mark=star, mark size = 3pt, color = cOS, dash dot},
    sos_line/.style={mark=halfcircle, mark size = 1.5pt, color = cSOS},
    sqr_line/.style={mark=-, mark size=3pt, color = cSQR},
    bs5_line/.style={mark=pentagon, mark size = 3pt, color = cSVD, dash dot dot},
    bs50_line/.style={mark=*, mark size = 2pt, color = c500},
    bs100_line/.style={mark=diamond, mark size = 3pt, color = orange, dashed},
    bs500_line/.style={mark=triangle, mark size = 3pt, color = black, densely dotted},
    }
}
% Add a serial/Oxford comma by default.
\newcommand{\creflastconjunction}{, and~}

% Used for creating new theorem and remark environments
\newsiamremark{remark}{Remark}
\newsiamremark{hypothesis}{Hypothesis}
\crefname{hypothesis}{Hypothesis}{Hypotheses}
\newsiamthm{example}{Example}
\crefname{example}{example}{examples}
\newsiamthm{claim}{Claim}

% Sets running headers as well as PDF title and authors
\headers{Iterative CUR}{N. Pritchard, T. Park, Y. Nakatsukasa, and P.G. Martinsson}

% Title. If the supplement option is on, then "Supplementary Material"
% is automatically inserted before the title.
\title{ Fast  Rank Adaptive CUR via a Recycled Small Sketch\thanks{Submitted to the editors September 26, 2025.
\funding{The work reported was supported by the Office of Naval Research (N00014-
18-1-2354), by the National Science Foundation (DMS-2313434),
and by the Department of Energy ASCR (DE-SC0025312). For the purpose of open access, the authors have applied a CC BY public copyright license to any author accepted manuscript arising from this submission.}}}

% Authors: full names plus addresses.
\author{Nathaniel Pritchard\thanks{Mathematical Institute at the University of Oxford.}
\and Taejun Park\thanks{Institute of Mathematics at EPF Lausanne.}
\and Yuji Nakatsukasa\footnotemark[2]
\and Per-Gunnar Martinsson\thanks{Department of Mathematics and the Oden Institute at the University of Texas at Austin.} 
}

\usepackage{amsopn}
\DeclareMathOperator{\diag}{diag}

%% Added on Overleaf: enabling xr
\makeatletter
\newcommand*{\addFileDependency}[1]{% argument=file name and extension
  \typeout{(#1)}% latexmk will find this if $recorder=0 (however, in that case, it will ignore #1 if it is a .aux or .pdf file etc and it exists! if it doesn't exist, it will appear in the list of dependents regardless)
  \@addtofilelist{#1}% if you want it to appear in \listfiles, not really necessary and latexmk doesn't use this
  \IfFileExists{#1}{}{\typeout{No file #1.}}% latexmk will find this message if #1 doesn't exist (yet)
}
\makeatother

\newcommand*{\myexternaldocument}[1]{%
    \externaldocument{#1}%
    \addFileDependency{#1.tex}%
    \addFileDependency{#1.aux}%
}
%%% END HELPER CODE
%%% Local Variables: 
%%% mode:latex
%%% TeX-master: "ex_article"
%%% End: 
